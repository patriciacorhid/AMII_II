\documentclass{article}
\usepackage[left=3cm,right=3cm,top=2cm,bottom=2cm]{geometry} % page settings
\usepackage{amsmath} % provides many mathematical environments & tools
\usepackage{amssymb}
\usepackage{amsfonts}
\usepackage[spanish]{babel}

\usepackage{multirow}

\usepackage{algorithm}
\usepackage{algpseudocode}
\usepackage{pifont}

\usepackage[utf8]{inputenc}
\setlength{\parindent}{0mm}

\usepackage[parfill]{parskip}

% Para el código
\usepackage{listings}
\usepackage{xcolor}
\definecolor{gray}{rgb}{0.5,0.5,0.5}
\newcommand{\n}[1]{{\color{gray}#1}}
\lstset{numbers=left,numberstyle=\small\color{gray}}

% Entorno para estilo de ejercicios
\newenvironment{ejercicio}[1]{\textbf{#1} \vspace*{5mm}}{\vspace*{5mm}}
\setlength{\parindent}{10pt} 

\begin{document}

\title{Problema Análisis Matemático II}
\author{Patricia Córdoba Hidalgo}
\date{\today}

\maketitle

Estudiad la convergencia puntual y uniforme de la sucesión de funciones $f_n$ definidas en $[0,1]$ mediante $$f_n(x) = x-x^n$$ para todo $x\in [0,1]$.

\textbf{Convergencia puntual:}

Sea $f_n(x) = x - x^n \text{ }\forall x \in [0,1] $, la función f a la que converge puntualmente ${f_n}$ es:

$f=\begin{cases}
 & \text{x si } x \in [0,1) \\
 & \text{0 si } x = 1 \\
\end{cases}$

Demostración:

$x=0 \Rightarrow 0 - 0^n = 1 \text{ }\forall n \in \mathbb{N} \\$

$x=1 \Rightarrow 1 - 1^n = 1 \text{ }\forall n \in \mathbb{N} \\$

$x \in (0,1) \Rightarrow  \lim\limits_{n\to\infty}x - x^n = \lim\limits_{n\to\infty}x - \lim\limits_{n\to\infty}x^n \\$

Como $x \in (0,1)$, $\lim\limits_{n\to\infty}x = x$ \text{   }y $\lim\limits_{n\to\infty}x^n=0$, luego $\lim\limits_{n\to\infty}x - x^n = x  \text{ }\forall x \in (0,1)$

\textbf{Convergencia uniforme:}

Como converge puntualmente, y que converja uniformente implica que converja puntualmente, el único candidato a límite de ${f_n}$ es f. Sin embargo, por un teorema, sabemos que ${f_n}$ converge uniformemente y $f_n$ es continua $\Rightarrow f$ es continua. Como f no es continua y $f_n$ es continua, entonces ${f_n}$ no puede converger uniformemente.

\end{document}